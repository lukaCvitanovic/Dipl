% -------------------------------------------------------------------
% napravili:
% Author: Toni Perković, toperkov@fesb.hr, toperkov@unist.hr
% Author: Marin Bugarić, mbugaric@fesb.hr
% Author: Ivo Stančić, istancic@fesb.hr
% modificirao: Ivan Tomac, itomac@fesb.hr
% FESB 2017;
% -------------------------------------------------------------------

%sazetak na hrvatskom jeziku
\newpage
\setlength{\parindent}{0in}
{\fontsize{14}{18}\bf {SAŽETAK}}

\vskip 15mm
\addcontentsline{toc}{chapter}{SAŽETAK}
   
\textnormal{
Tema diplomskog rada je pronalazak front-end ranjivosti web aplikacija, načini na koje se te ranjivosti mogu zloupotrjebiti te koje su metode obrane od napada koji iskorištavaju pronađene ranjivosti. }

\textnormal{
XSS ranjivost preko upravitelja lozinki opisuje napad koji iskorištava XSS ranjivost u blog stranici. Ranjivost koju web stranica posjeduje je mogućnost da korisnik upiše HTML kod u tekst članka koji je moguće iskoristi loše konfiguriranji upravitelj lozinki i na taj način doći do korisničko imena i lozinke žrtve.}

\textnormal{
U sljedečem napadu je iskorištena DOM clobbering ranjivost blog web stranice. Ovaj put web stranica primjernju pročišćavanje korisničkog koda te dopušta sigurne HTML elemente, atribute i mrežne protokole. Mđutim, ova stranica ima ranjivost u načinu na koji je kod napisan te propust u filteru koji je se primjenjuje na tekst blog članka. Ovo zajedno omogućava napadču izvršavanje bilo kakvog koda unutar web stranice.}

\textnormal{
Slijpi SQLi napad iskorištava dobro napravljenu web aplikaciju koja ne reagira na kašnjenja ili greške izazvane na bazi podataka. Ovaj napad iskorištava zadanu konfiguraciju baze podataka kako bi se gotovo neprimjetno izvukli podatci.}

\textnormal{
Ranjivost koja je pronađena u ASP .NET-u omogućava napdaču izvršavanje proizvoljnog koda, jer se zloupotrjebljava zastarjela funkcionalnost. Ovaj napad se izvodi kreiranjem URL-a kojim se može u web stranciu učitati maliciozna skripta, jer se koristi zastarjela funkcionalnost.}

\textnormal{
I na kraju je opisana ranjivost PHP servera. Ova ranjivost se nalazi u načinu na koji PHP server interpretira relativne URL adrese. Ova ranjivost zahtjeva korištenje Internet Explorera koji u konačnici omogućava primjenu CSS keyloggera kojim je moguća izvuči bilo koji tajni ili skriveni podataka s web stranice.}


\vskip 15mm
\bf{Ključne riječi:\\}
\textnormal{XSS, SQLi, Dom clobbering, ASP. NET, PHP, CSS keylogger, hakerski napad, ranjivosti}


