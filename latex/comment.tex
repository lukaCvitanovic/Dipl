\documentclass[parskip=false, DIV=8, headings=normal, pagesize=auto]{artikel3}%{scrartcl}

\usepackage{fixltx2e}
\usepackage{etex}
\usepackage{xspace}
\usepackage{lmodern}
\usepackage[T1]{fontenc}
\usepackage{textcomp}
\usepackage{microtype}
\usepackage[unicode=true]{hyperref}

\newenvironment*{noverb}{%
  \flushleft
  \vskip\parskip
  \parskip=0pt\relax
}{%
  \endflushleft
}

\newcommand*{\mail}[1]{\href{mailto:#1}{\texttt{#1}}}
\newcommand*{\pkg}[1]{\textsf{#1}}
\newcommand*{\cs}[1]{\texttt{\textbackslash#1}}
\makeatletter
\newcommand*{\cmd}[1]{\cs{\expandafter\@gobble\string#1}}
\makeatother
\newcommand*{\env}[1]{\texttt{#1}}
\newcommand*{\meta}[1]{\textlangle\textsl{#1}\textrangle}
\newcommand*{\marg}[1]{\texttt{\{}\meta{#1}\texttt{\}}}

%\addtokomafont{title}{\rmfamily}

\title{The \pkg{comment} package\thanks{This manual corresponds to
    \pkg{comment}~v3.8 of July 2016.}}
%
\author{Victor Eijkhout\\\mail{victor@eijkhout.net}} \date{August 2016}


\begin{document}

\maketitle

\section{Purpose:}

Selectively in/exclude pieces of text: the user can define new
comment versions, and each is controlled separately.
Special comments can be defined where the user specifies the
action that is to be taken with each comment line.

Plain \TeX\ support has been phased out.

As of 3.8 the package will increasingly use e\TeX\ features, for
instance to solve Unicode support issues.

\section{Usage:}

The `\env{comment}' environment is defined by default:
all text included between
%
\begin{verbatim}
\begin{comment}
 ... 
\end{comment}
\end{verbatim}
%
is discarded. 

The opening and closing commands should appear on a line
of their own. No starting spaces, nothing after it.
This environment should work with arbitrary amounts
of comment, and the comment can be arbitrary text.

Other `\env{comment}' environments are defined by
and are selected/deselected with
%
\begin{verbatim}
\includecomment{versiona}
\excludecoment{versionb}
\end{verbatim}
%
These environments are used as
%
\begin{verbatim}
\begin{versiona}
 ... 
\end{versiona}
\end{verbatim}
%
with the opening and closing commands again on a line of 
their own.

Note: for an included comment, the
\cmd{\begin} and \cmd{\end} lines act as if they don't exist.
In particular, they don't imply grouping, so assignments 
\&c are not local.

Trick for short in/exclude macros (such as \verb+\maybe{this snippet}+):
%
\begin{verbatim}
\includecomment{cond}
\newcommand{\maybe}[1]{}
\begin{cond}
\renewcommand{\maybe}[1]{#1}
\end{cond}
\end{verbatim}

\section{Special comments}

It is possible to make highly customized versions of the comment environment.
Special comments are defined as
%
\begin{noverb}
\cmd{\specialcomment}\marg{name}\marg{before commands}\marg{after commands}
\end{noverb}
%
where the second and third arguments are executed before
and after each comment block. You can use this for global
formatting commands.

To keep definitions \&c local, you can include \cmd{\begingroup}
in the `\meta{before commands}' and \cmd{\endgroup} in the `\meta{after commands}'.

Example:
%
\begin{verbatim}
\specialcomment{smalltt}
    {\begingroup\ttfamily\footnotesize}{\endgroup}
\end{verbatim}
%
Special comments are automatically included.

The comment environments use two auxiliary commands. You can get
nifty special effects by redefining them.

\subsection{The cutfile}

The commented text is written to an external file, the `cutfile'. Default definition:
\begin{verbatim}
  \def\CommentCutFile{comment.cut}
\end{verbatim}

Included comments are processed like this:
\begin{verbatim}
  \def\ProcessCutFile{\input{\CommentCutFile}\relax}
\end{verbatim}
  and excluded files have
\begin{verbatim}
  \def\ProcessCutFile{}
\end{verbatim}

\begin{itemize}
\item By redefining the name of the cutfile, the value of the macro
  \cmd{CommentCutFile}, it becomes possible to have nested comment environments.
\item If you are writing a textbook, you could have the answers to
  exercises in your source, but write them to file rather than
  formatting them:
\begin{verbatim}
\generalcomment{answer}
  {\begingroup
   \edef\tmp{\def\noexpand\CommentCutFile
                 {answers/\chapshortname-an\noexpand\arabic{excounter}.tex}}\tmp
   \def\ProcessCutFile{}}
  {\ifIncludeAnswers \begin{quote}
   \leavevmode
   \hbox{\kern-\unitindent 
         \textbf Solution to exercise \arabic{chapter}.\arabic{excounter}.\hspace{1em}}%
     \ignorespaces\it
   \input{\CommentCutFile}
   \end{quote}\fi
   \endgroup}
\end{verbatim}
\end{itemize}

\subsection{Comment inclusion}

The inclusion of the comment is done
by \cmd{\ProcessCutFile}, so you can redefine that:
\begin{verbatim}
  \specialcomment
    {mathexamplewithcode}
    {\begingroup\def\ProcessCutFile{}} % arg1
    {\verbatiminput{\CommentCutFile}   % arg2
     \endgroup
     This gives:
     \begin{equation} \input{\CommentCutFile} \end{equation}
    }
\end{verbatim}
The idea here is to disable inclusion of the file,
but include it in the after commands, in display math.

\subsection{Processing each line}

You can also apply processing to each line.
By defining a control sequence 
\begin{verbatim}
  \def\Thiscomment##1{...}
\end{verbatim}
in the before commands the user can
specify what is to be done with each comment line. If something 
needs to be written to file, use \verb+\WriteCommentLine{the stuff}+
Example:
\begin{verbatim}
  \specialcomment
    {underlinecomment}
    {\def\ThisComment##1{\WriteCommentLine{\underline{##1}\par}}
     \par}
    {\par}
\end{verbatim}

\subsection{More examples}

\begin{verbatim}
\newcount\comlines
\specialcomment{countedcomment}
 {\comlines=0\relax \def\ProcessCutFile{}%
  \def\ThisComment##1{\global\advance\comlines1\relax}}
 {**Comment: \number\comlines\ line(s) removed**}
\end{verbatim}

\begin{verbatim}
\specialcomment
    {underlinecomment}
    {%
      \def\ProcessCutFile{\input{\CommentCutFile}\relax}
      \def\ThisComment##1{\WriteCommentLine{u: \underline{##1}\par}}
      \par
    }
    {\par}
\end{verbatim}

\section{Unicode support}

Unicode support works if you use e\TeX, which is for instance the case
if you use pdflatex. You need the following lines:
\begin{verbatim}
\usepackage[T1]{fontenc}
\usepackage[utf8]{inputenc}
\end{verbatim}
in your preamble.

\section{Change log}

\subsection{Changes in version 3.1}

\begin{itemize}
\item updated author's address
\item cleaned up some code
\item trailing contents on \cmd{\begin}\marg{env} line is always discarded
    even if you've done \cmd{\includecomment}\marg{env}
\item comments no longer define grouping!! you can even
  % 
\begin{verbatim}
\includecomment{env}
\begin{env}
\begin{itemize}
\end{env}
\end{verbatim}
  % 
  Isn't that something\ldots
\item included comments are written to file and input again.
\end{itemize}


\subsection{Changes in 3.2}

\begin{itemize}
\item \cmd{\specialcomment} brought up to date (thanks to Ivo Welch).
\end{itemize}


\subsection{Changes in 3.3}

\begin{itemize}
\item updated author's address again
\item parametrised \cmd{\CommentCutFile}
\end{itemize}


\subsection{Changes in 3.4}

\begin{itemize}
\item added GNU public license
\item added \cmd{\processcomment}, because Ivo's fix (above) brought an
  inconsistency to light.
\end{itemize}

  
\subsection{Changes in 3.5}

\begin{itemize}
\item corrected typo in header.
\item changed author email
\item corrected \cmd{\specialcomment} yet again.
\item fixed excludecomment of an earlier defined environment.
\end{itemize}


\subsection{Changes in 3.6}

\begin{itemize}
\item The `cut' file is now written more verbatim, using \cmd{\meaning};
  some people reported having trouble with ISO~latin~1, or \texttt{umlaute.sty}.
\item removed some \cmd{\newif} statements.
  Has this suddenly become \cmd{\outer} again?
\end{itemize}


\subsection{Changes in 3.8}

T1 font encoding is now supported. See t1test.tex.

\end{document}
