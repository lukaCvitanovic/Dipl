% -------------------------------------------------------------------
% napravili:
% Author: Toni Perković, toperkov@fesb.hr, toperkov@unist.hr
% Author: Marin Bugarić, mbugaric@fesb.hr
% Author: Ivo Stančić, istancic@fesb.hr
% modificirao: Ivan Tomac, itomac@fesb.hr
% FESB 2017;
% -------------------------------------------------------------------

%sazetak na engleskom jeziku

\newpage
\setlength{\parindent}{0in}

\begin{center}
	{\LARGE\bf{FRONT-END VULNERABILITIES OF WEB APPLICATIONS}}
\end{center}
\vskip 10mm

{\fontsize{14}{18}\bf {SUMMARY}}

\vskip 15mm
\addcontentsline{toc}{chapter}{SUMMARY}
\textnormal{
The theme of this is the discovery of front-end vulnerabilities, the way these vulnerabilities could be exploited with cyber attacks and defense methods that could be used against these attacks.}

\textnormal{
XSS vulnerability with password manager describes the attack that exploits XSS vulnerability found in blog web site. This vulnerability allows the users to write HTML code, into the blog post, which exploits badly configured password manager to get the victims user name and password.}

\textnormal{
The next attack exploits DOM clobbering vulnerability in the blog web site, but this time the web site is performing purification of user generated code and allows the use of safe HTML elements, attributes and protocols. However, this web site has the vulnerability in the way it is coded and in the text filter that is being applied to the user text. Together this allows the execution of malicious inside the web site.}

\textnormal{
Blind SQLi attack exploits a well made web application that does not change even when there are errors or time delays being created on the database. This attack exploits the default configuration of the database to extract the data.}

\textnormal{
Vulnerability that was found in ASP .NET framework allows the attacker to execute what ever code he wants, all because the framework supports outdated functionality. This attack is done by creating the URL that exploits the old functionality, which intern allows the execution of malicious script.}

\textnormal{
The last attack describes the PHP server vulnerability. This vulnerability lies in a way the PHP server interprets relative URL addresses. The described vulnerability in combination with Internet Explorer allows the deployment of CSS keylogger which can extract any secret or hidden data on the web site.}

\bf{Keywords  :\\}
\textnormal{XSS , SQLi, DOM clobbering, ASP. NET, PHP, CSS keylogger, cyber attack, vulnerabilities}
